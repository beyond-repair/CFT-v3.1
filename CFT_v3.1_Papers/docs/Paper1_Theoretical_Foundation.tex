\documentclass[12pt, a4paper]{article}
\usepackage{amsmath, amssymb, amsfonts}
\usepackage{graphicx}
\usepackage{geometry}
\usepackage{hyperref}
\geometry{margin=1in}

\title{The Screened Metric: A Volumetric Theory of Gravitation}
\author{William B. Ware}
\date{\today}

\begin{document}

\maketitle

\begin{center}
\textbf{Copyright (c) 2025 William B. Ware. CC BY 4.0 License.}
\end{center}

\begin{abstract}
This paper introduces the theoretical foundation of Coherence Field Theory (CFT) v3.1, proposing that gravity is a volumetric metric response of the vacuum. A fundamental elasticity constant, the Ware Constant (\(\zeta = 0.08\)), is activated in ordered, virialized systems via the Mass-Density Screening (MDS) mechanism. This framework eliminates the need for hypothetical dark matter particles and provides a unified, scalable metric across atomic to galactic systems.
\end{abstract}

\section{Introduction}
Gravity is not merely a surface interaction of mass; it is a volumetric response encoded in the structure of spacetime itself. The vacuum possesses an intrinsic elasticity, suppressed in disordered systems and triggered by structural order.

\section{Core Discovery}
\begin{itemize}
    \item \textbf{Volumetric Metric Response:} The vacuum contributes to gravitational effects proportionally to structural order.  
    \item \textbf{Ware Constant (\(\zeta = 0.08\)):} Universal elasticity of the vacuum.  
    \item \textbf{Mass-Density Screening (MDS):} Activation function \(S(\rho, L)\) ensures effects manifest only in virialized, ordered baryonic systems.
\end{itemize}

\section{Field Equation}
The master field equation of CFT v3.1 extends Einstein's general relativity:

\begin{equation}
G_{\mu\nu} + \Lambda g_{\mu\nu} = 8\pi G \Big( T_{\mu\nu} + S(\rho, L) \cdot \zeta I_{\mu\nu} \Big)
\end{equation}

Where:
\begin{itemize}
    \item \(I_{\mu\nu}\) = Coherence Tensor (structural information density)  
    \item \(S(\rho, L)\) = Screening factor (activation threshold)  
    \item \(\zeta = 0.08\) = Ware Constant  
\end{itemize}

\section{Significance}
This equation replaces the need for dark matter particles with a **geometric backreaction of the vacuum itself**, producing measurable effects in galaxies, clusters, and atomic systems where the density and scale thresholds are met.

\section{Conclusion}
The theoretical framework presented here is mathematically closed, universal, and parsimonious. It provides the foundation for cross-scale empirical verification presented in Paper 2.

\end{document}
